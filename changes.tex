% changes.tex
% Copyright (C) 2012  Michael Chang
% based on
% uw-wkrpt-se.tex - An example work report that uses uw-wkrpt.cls
% Copyright (C) 2002,2003  Simon Law
% 
% This program is free software; you can redistribute it and/or modify
% it under the terms of the GNU General Public License as published by
% the Free Software Foundation; either version 2 of the License, or
% (at your option) any later version.
% 
% This program is distributed in the hope that it will be useful,
% but WITHOUT ANY WARRANTY; without even the implied warranty of
% MERCHANTABILITY or FITNESS FOR A PARTICULAR PURPOSE.  See the
% GNU General Public License for more details.
% 
% You should have received a copy of the GNU General Public License
% along with this program; if not, write to the Free Software
% Foundation, Inc., 59 Temple Place, Suite 330, Boston, MA  02111-1307  USA
%
%%%%%%%%%%%%%%%%%%%%%%%%%%%%%%%%%%%%%%%%%%%%%%%%%%%%%%%%%%%%%%%%%%%%%
%
% We begin by calling the workreport class which includes all the
% definitions for the macros we will use.
\documentclass[se,resubmit]{uw-wkrpt}

% LaTeX preamble: load some packages to add functionality
\usepackage{graphicx} % Include graphic importing

\usepackage[T1]{fontenc} % Better fonts
\usepackage{ae,aecompl}

\usepackage{indentfirst} % Indent first paragraph of each section

% For mathematical symbols in our pseudocode
\usepackage{amsmath}

% Use the algorithmicx package for pseudocode
\usepackage{algorithm}
\usepackage{algpseudocode}

\usepackage{listings}

% This needs to be the last package loaded
\usepackage[pdftex]{hyperref} % Generate PDF links and bookmarks.
\hypersetup{
  bookmarks=true,
  bookmarksnumbered=true
}

% Now we will begin writing the document.
\begin{document}

%%%%%%%%%%%%%%%%%%%%%%%%%%%%%%%%%%%%%%%%%%%%%%%%%%%%%%%%%%%%%%%%%%%%%
%% List of Changes
%%%%%%%%%%%%%%%%%%%%%%%%%%%%%%%%%%%%%%%%%%%%%%%%%%%%%%%%%%%%%%%%%%%%%
%% \main will make the \section commands numbered again,
%% it will also use arabic page numbers.
\mainmatter

% You must have an Introduction
\section{List of Changes}\label{sec:changes}

\begin{itemize}
  \item Use \LaTeX (instead of Microsoft Word) to typeset the report, using
	the \texttt{uw-wkrpt.cls} class by Simon Law, et. al. (The class
	was modified to note that the report is a resubmission in the
	letter of submittal; changes can be found at
	\url{https://github.com/cbhl/uw-wkrpt-resubmit}.) This should
	resolve most of the issues in "general formatting".
  \item Change the title to be more descriptive.
  \item Fix the name and location of the employer on the cover page.
        (YouTube, LLC is owned by Google Inc.)
  \item Update the report date.
  \item Make the letter of submittal more like the sample letter.
  \item Insert an Executive Summary before the ToC.
  \item Merge 'Background' and 'Problem' to two subsections of a section
        called 'Introduction'.
  \item Replace 'Justification' with an 'Analysis' section, detailing
        criteria and alternatives, placing it before the recommendation.
  \item Insert a 'Conclusion' section after the new 'Analysis' section and
        before the 'Recommendations' section.
  \item Replace figure 'Preview Request Flow' with 'Swimlane Activity
        Diagram'.
  \item New figure: 'Network Diagram' to show domain and environment.
  \item New figure: 'Sample Edit List' to provide context.
  \item New figures: 'Positionable/Trimmable Audio - User Interface',
        'Positioning Audio', 'Trimming Audio' to provide context.
  \item Add a label to the table in Appendix A.
  \item Provide more context in the problem statement.
  \item Look at more plausible alternatives than the ones considered in the
        original report.
  \item Infer a recommendation based on the assumptions and the results of
        the analysis (as opposed to just making a straightforward, but
	possibly inaccurate recommendation).
\end{itemize}

\end{document}
